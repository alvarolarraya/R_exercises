% Options for packages loaded elsewhere
\PassOptionsToPackage{unicode}{hyperref}
\PassOptionsToPackage{hyphens}{url}
%
\documentclass[
]{article}
\usepackage{lmodern}
\usepackage{amssymb,amsmath}
\usepackage{ifxetex,ifluatex}
\ifnum 0\ifxetex 1\fi\ifluatex 1\fi=0 % if pdftex
  \usepackage[T1]{fontenc}
  \usepackage[utf8]{inputenc}
  \usepackage{textcomp} % provide euro and other symbols
\else % if luatex or xetex
  \usepackage{unicode-math}
  \defaultfontfeatures{Scale=MatchLowercase}
  \defaultfontfeatures[\rmfamily]{Ligatures=TeX,Scale=1}
\fi
% Use upquote if available, for straight quotes in verbatim environments
\IfFileExists{upquote.sty}{\usepackage{upquote}}{}
\IfFileExists{microtype.sty}{% use microtype if available
  \usepackage[]{microtype}
  \UseMicrotypeSet[protrusion]{basicmath} % disable protrusion for tt fonts
}{}
\makeatletter
\@ifundefined{KOMAClassName}{% if non-KOMA class
  \IfFileExists{parskip.sty}{%
    \usepackage{parskip}
  }{% else
    \setlength{\parindent}{0pt}
    \setlength{\parskip}{6pt plus 2pt minus 1pt}}
}{% if KOMA class
  \KOMAoptions{parskip=half}}
\makeatother
\usepackage{xcolor}
\IfFileExists{xurl.sty}{\usepackage{xurl}}{} % add URL line breaks if available
\IfFileExists{bookmark.sty}{\usepackage{bookmark}}{\usepackage{hyperref}}
\hypersetup{
  hidelinks,
  pdfcreator={LaTeX via pandoc}}
\urlstyle{same} % disable monospaced font for URLs
\usepackage[margin=1in]{geometry}
\usepackage{color}
\usepackage{fancyvrb}
\newcommand{\VerbBar}{|}
\newcommand{\VERB}{\Verb[commandchars=\\\{\}]}
\DefineVerbatimEnvironment{Highlighting}{Verbatim}{commandchars=\\\{\}}
% Add ',fontsize=\small' for more characters per line
\usepackage{framed}
\definecolor{shadecolor}{RGB}{248,248,248}
\newenvironment{Shaded}{\begin{snugshade}}{\end{snugshade}}
\newcommand{\AlertTok}[1]{\textcolor[rgb]{0.94,0.16,0.16}{#1}}
\newcommand{\AnnotationTok}[1]{\textcolor[rgb]{0.56,0.35,0.01}{\textbf{\textit{#1}}}}
\newcommand{\AttributeTok}[1]{\textcolor[rgb]{0.77,0.63,0.00}{#1}}
\newcommand{\BaseNTok}[1]{\textcolor[rgb]{0.00,0.00,0.81}{#1}}
\newcommand{\BuiltInTok}[1]{#1}
\newcommand{\CharTok}[1]{\textcolor[rgb]{0.31,0.60,0.02}{#1}}
\newcommand{\CommentTok}[1]{\textcolor[rgb]{0.56,0.35,0.01}{\textit{#1}}}
\newcommand{\CommentVarTok}[1]{\textcolor[rgb]{0.56,0.35,0.01}{\textbf{\textit{#1}}}}
\newcommand{\ConstantTok}[1]{\textcolor[rgb]{0.00,0.00,0.00}{#1}}
\newcommand{\ControlFlowTok}[1]{\textcolor[rgb]{0.13,0.29,0.53}{\textbf{#1}}}
\newcommand{\DataTypeTok}[1]{\textcolor[rgb]{0.13,0.29,0.53}{#1}}
\newcommand{\DecValTok}[1]{\textcolor[rgb]{0.00,0.00,0.81}{#1}}
\newcommand{\DocumentationTok}[1]{\textcolor[rgb]{0.56,0.35,0.01}{\textbf{\textit{#1}}}}
\newcommand{\ErrorTok}[1]{\textcolor[rgb]{0.64,0.00,0.00}{\textbf{#1}}}
\newcommand{\ExtensionTok}[1]{#1}
\newcommand{\FloatTok}[1]{\textcolor[rgb]{0.00,0.00,0.81}{#1}}
\newcommand{\FunctionTok}[1]{\textcolor[rgb]{0.00,0.00,0.00}{#1}}
\newcommand{\ImportTok}[1]{#1}
\newcommand{\InformationTok}[1]{\textcolor[rgb]{0.56,0.35,0.01}{\textbf{\textit{#1}}}}
\newcommand{\KeywordTok}[1]{\textcolor[rgb]{0.13,0.29,0.53}{\textbf{#1}}}
\newcommand{\NormalTok}[1]{#1}
\newcommand{\OperatorTok}[1]{\textcolor[rgb]{0.81,0.36,0.00}{\textbf{#1}}}
\newcommand{\OtherTok}[1]{\textcolor[rgb]{0.56,0.35,0.01}{#1}}
\newcommand{\PreprocessorTok}[1]{\textcolor[rgb]{0.56,0.35,0.01}{\textit{#1}}}
\newcommand{\RegionMarkerTok}[1]{#1}
\newcommand{\SpecialCharTok}[1]{\textcolor[rgb]{0.00,0.00,0.00}{#1}}
\newcommand{\SpecialStringTok}[1]{\textcolor[rgb]{0.31,0.60,0.02}{#1}}
\newcommand{\StringTok}[1]{\textcolor[rgb]{0.31,0.60,0.02}{#1}}
\newcommand{\VariableTok}[1]{\textcolor[rgb]{0.00,0.00,0.00}{#1}}
\newcommand{\VerbatimStringTok}[1]{\textcolor[rgb]{0.31,0.60,0.02}{#1}}
\newcommand{\WarningTok}[1]{\textcolor[rgb]{0.56,0.35,0.01}{\textbf{\textit{#1}}}}
\usepackage{graphicx,grffile}
\makeatletter
\def\maxwidth{\ifdim\Gin@nat@width>\linewidth\linewidth\else\Gin@nat@width\fi}
\def\maxheight{\ifdim\Gin@nat@height>\textheight\textheight\else\Gin@nat@height\fi}
\makeatother
% Scale images if necessary, so that they will not overflow the page
% margins by default, and it is still possible to overwrite the defaults
% using explicit options in \includegraphics[width, height, ...]{}
\setkeys{Gin}{width=\maxwidth,height=\maxheight,keepaspectratio}
% Set default figure placement to htbp
\makeatletter
\def\fps@figure{htbp}
\makeatother
\setlength{\emergencystretch}{3em} % prevent overfull lines
\providecommand{\tightlist}{%
  \setlength{\itemsep}{0pt}\setlength{\parskip}{0pt}}
\setcounter{secnumdepth}{-\maxdimen} % remove section numbering

\author{}
\date{\vspace{-2.5em}}

\begin{document}

\hypertarget{pruxe1ctica-3}{%
\section{Práctica 3}\label{pruxe1ctica-3}}

\hypertarget{uxe1lvaro-larraya}{%
\subsection{Álvaro Larraya}\label{uxe1lvaro-larraya}}

\hypertarget{sea-x-una-v.a-chi-cuadrado-con-8-grados-de-libertad}{%
\subsubsection{2.1) Sea X una v.a chi-cuadrado con 8 grados de
libertad}\label{sea-x-una-v.a-chi-cuadrado-con-8-grados-de-libertad}}

Representa gráficamente la función de densidad de X

\begin{Shaded}
\begin{Highlighting}[]
\NormalTok{x<-}\KeywordTok{seq}\NormalTok{(}\DecValTok{0}\NormalTok{,}\DecValTok{30}\NormalTok{,}\FloatTok{0.01}\NormalTok{)}
\KeywordTok{plot}\NormalTok{(x,}\KeywordTok{dchisq}\NormalTok{(x,}\DecValTok{8}\NormalTok{),}\DataTypeTok{type=}\StringTok{"l"}\NormalTok{,}\DataTypeTok{ylab=}\StringTok{"f(x)"}\NormalTok{,}\DataTypeTok{main=}\StringTok{"Chisq with 8 d.f."}\NormalTok{)}
\end{Highlighting}
\end{Shaded}

\includegraphics{Practica3_ÁlvaroLarraya_files/figure-latex/unnamed-chunk-1-1.pdf}

Calcula P(X \textless{} 3)

\begin{Shaded}
\begin{Highlighting}[]
\KeywordTok{pchisq}\NormalTok{(}\DecValTok{3}\NormalTok{,}\DecValTok{8}\NormalTok{)}
\end{Highlighting}
\end{Shaded}

\begin{verbatim}
## [1] 0.06564245
\end{verbatim}

P(3 ≤ X ≤ 5)

\begin{Shaded}
\begin{Highlighting}[]
\KeywordTok{pchisq}\NormalTok{(}\DecValTok{5}\NormalTok{,}\DecValTok{8}\NormalTok{)}\OperatorTok{-}\KeywordTok{pchisq}\NormalTok{(}\DecValTok{3}\NormalTok{,}\DecValTok{8}\NormalTok{)}
\end{Highlighting}
\end{Shaded}

\begin{verbatim}
## [1] 0.1767814
\end{verbatim}

Calcula a tal que P(X ≤ a) = 0.95

\begin{Shaded}
\begin{Highlighting}[]
\KeywordTok{qchisq}\NormalTok{(}\FloatTok{0.95}\NormalTok{,}\DecValTok{8}\NormalTok{)}
\end{Highlighting}
\end{Shaded}

\begin{verbatim}
## [1] 15.50731
\end{verbatim}

Calcula la mediana

\begin{Shaded}
\begin{Highlighting}[]
\KeywordTok{qchisq}\NormalTok{(}\FloatTok{0.5}\NormalTok{,}\DecValTok{8}\NormalTok{)}
\end{Highlighting}
\end{Shaded}

\begin{verbatim}
## [1] 7.344121
\end{verbatim}

\hypertarget{sea-x-una-v.a.-t10}{%
\subsubsection{2.2) Sea X una v.a. t10}\label{sea-x-una-v.a.-t10}}

Representa gráficamente la función de densidad de X

\begin{Shaded}
\begin{Highlighting}[]
\NormalTok{x<-}\KeywordTok{seq}\NormalTok{(}\OperatorTok{-}\DecValTok{4}\NormalTok{,}\DecValTok{4}\NormalTok{,}\FloatTok{0.01}\NormalTok{)}
\KeywordTok{plot}\NormalTok{(x,}\KeywordTok{dt}\NormalTok{(x,}\DecValTok{10}\NormalTok{),}\DataTypeTok{type=}\StringTok{"l"}\NormalTok{,}\DataTypeTok{ylab=}\StringTok{"f(x)"}\NormalTok{,}\DataTypeTok{main=}\StringTok{"student t with 10 d.f."}\NormalTok{)}
\end{Highlighting}
\end{Shaded}

\includegraphics{Practica3_ÁlvaroLarraya_files/figure-latex/unnamed-chunk-6-1.pdf}

Calcula P(X \textless{} 2)

\begin{Shaded}
\begin{Highlighting}[]
\KeywordTok{pt}\NormalTok{(}\DecValTok{2}\NormalTok{, }\DecValTok{10}\NormalTok{)}
\end{Highlighting}
\end{Shaded}

\begin{verbatim}
## [1] 0.963306
\end{verbatim}

P(−0.5 ≤ X ≤ 0.5)

\begin{Shaded}
\begin{Highlighting}[]
\KeywordTok{pt}\NormalTok{(}\FloatTok{0.5}\NormalTok{, }\DecValTok{10}\NormalTok{)}\OperatorTok{-}\KeywordTok{pt}\NormalTok{(}\OperatorTok{-}\FloatTok{0.5}\NormalTok{, }\DecValTok{10}\NormalTok{)}
\end{Highlighting}
\end{Shaded}

\begin{verbatim}
## [1] 0.3721064
\end{verbatim}

Calcula a tal que P(−1 − a ≤ X ≤ 1 + a) = 0.80

\begin{Shaded}
\begin{Highlighting}[]
\KeywordTok{qt}\NormalTok{(}\FloatTok{0.8}\NormalTok{,}\DecValTok{10}\NormalTok{)}
\end{Highlighting}
\end{Shaded}

\begin{verbatim}
## [1] 0.8790578
\end{verbatim}

\hypertarget{sea-x-una-v.a.-f-25}{%
\subsubsection{2.3) Sea X una v.a. F 2,5}\label{sea-x-una-v.a.-f-25}}

Representa gráficamente la función de densidad de X

\begin{Shaded}
\begin{Highlighting}[]
\NormalTok{x<-}\KeywordTok{seq}\NormalTok{(}\DecValTok{0}\NormalTok{,}\DecValTok{10}\NormalTok{,}\FloatTok{0.01}\NormalTok{)}
\KeywordTok{plot}\NormalTok{(x,}\KeywordTok{df}\NormalTok{(x,}\DecValTok{2}\NormalTok{,}\DecValTok{5}\NormalTok{),}\DataTypeTok{type=}\StringTok{"l"}\NormalTok{,}\DataTypeTok{main=}\StringTok{"F with 2 and 5 d.f."}\NormalTok{)}
\end{Highlighting}
\end{Shaded}

\includegraphics{Practica3_ÁlvaroLarraya_files/figure-latex/unnamed-chunk-10-1.pdf}

Calcula P(X \textgreater{} 2)

\begin{Shaded}
\begin{Highlighting}[]
\DecValTok{1}\OperatorTok{-}\KeywordTok{pf}\NormalTok{(}\DecValTok{2}\NormalTok{,}\DecValTok{2}\NormalTok{,}\DecValTok{5}\NormalTok{)}
\end{Highlighting}
\end{Shaded}

\begin{verbatim}
## [1] 0.2300481
\end{verbatim}

P(1 ≤ X ≤ 5)

\begin{Shaded}
\begin{Highlighting}[]
\KeywordTok{pf}\NormalTok{(}\DecValTok{5}\NormalTok{,}\DecValTok{2}\NormalTok{,}\DecValTok{5}\NormalTok{)}\OperatorTok{-}\KeywordTok{pf}\NormalTok{(}\DecValTok{1}\NormalTok{,}\DecValTok{2}\NormalTok{,}\DecValTok{5}\NormalTok{)}
\end{Highlighting}
\end{Shaded}

\begin{verbatim}
## [1] 0.3670511
\end{verbatim}

Calcula el 90 percentil

\begin{Shaded}
\begin{Highlighting}[]
\KeywordTok{qf}\NormalTok{(}\FloatTok{0.9}\NormalTok{,}\DecValTok{2}\NormalTok{,}\DecValTok{5}\NormalTok{)}
\end{Highlighting}
\end{Shaded}

\begin{verbatim}
## [1] 3.779716
\end{verbatim}

\hypertarget{simula-m-20000-muestra-de-tamauxf1o-n-2-de-una-distribuciuxf3n}{%
\subsubsection{2.4) Simula m = 20000 muestra de tamaño n = 2 de una
distribución}\label{simula-m-20000-muestra-de-tamauxf1o-n-2-de-una-distribuciuxf3n}}

Exp(λ = 4). Para cada muestra, calcula la suma de cada muestra. Fija la
semilla aleatoria a 235

\begin{Shaded}
\begin{Highlighting}[]
\KeywordTok{set.seed}\NormalTok{(}\DecValTok{235}\NormalTok{)}
\NormalTok{m<-}\DecValTok{20000}
\NormalTok{n<-}\DecValTok{2}
\NormalTok{sum.exp <-}\StringTok{ }\KeywordTok{numeric}\NormalTok{(m)}
\ControlFlowTok{for}\NormalTok{(i }\ControlFlowTok{in} \DecValTok{1}\OperatorTok{:}\NormalTok{m)\{}
\NormalTok{samp <-}\StringTok{ }\KeywordTok{rexp}\NormalTok{(n,}\DecValTok{4}\NormalTok{)}
\NormalTok{sum.exp[i] <-}\StringTok{ }\KeywordTok{sum}\NormalTok{(samp)}
\NormalTok{\}}
\end{Highlighting}
\end{Shaded}

Representa el histograma de frecuencias relativas

\begin{Shaded}
\begin{Highlighting}[]
\KeywordTok{hist}\NormalTok{(sum.exp,}\DataTypeTok{prob=}\OtherTok{TRUE}\NormalTok{, }\DataTypeTok{main=}\StringTok{"Suma de muestras"}\NormalTok{)}
\NormalTok{x <-}\StringTok{ }\KeywordTok{seq}\NormalTok{(}\DecValTok{0}\NormalTok{,}\DecValTok{10}\NormalTok{,}\FloatTok{0.01}\NormalTok{)}
\NormalTok{y <-}\StringTok{ }\KeywordTok{dexp}\NormalTok{(x,}\DecValTok{4}\NormalTok{)}
\KeywordTok{lines}\NormalTok{(x,y)}
\end{Highlighting}
\end{Shaded}

\includegraphics{Practica3_ÁlvaroLarraya_files/figure-latex/unnamed-chunk-15-1.pdf}

Añade la función de densidad de una Γ(α = 2, λ = 4)

\begin{Shaded}
\begin{Highlighting}[]
\NormalTok{x<-}\KeywordTok{seq}\NormalTok{(}\DecValTok{0}\NormalTok{,}\DecValTok{30}\NormalTok{,}\FloatTok{0.01}\NormalTok{)}
\KeywordTok{plot}\NormalTok{(x,}\KeywordTok{dgamma}\NormalTok{(x,}\DecValTok{2}\NormalTok{,}\DecValTok{4}\NormalTok{),}\DataTypeTok{type=}\StringTok{"l"}\NormalTok{,}\DataTypeTok{ylab=}\StringTok{"f(x)"}\NormalTok{,}\DataTypeTok{main=}\StringTok{"Γ(α = 2, λ = 4)"}\NormalTok{)}
\end{Highlighting}
\end{Shaded}

\includegraphics{Practica3_ÁlvaroLarraya_files/figure-latex/unnamed-chunk-16-1.pdf}

Y una N(2/4,√2/4)

\begin{Shaded}
\begin{Highlighting}[]
\NormalTok{x<-}\KeywordTok{seq}\NormalTok{(}\OperatorTok{-}\DecValTok{3}\NormalTok{,}\DecValTok{4}\NormalTok{,}\FloatTok{0.01}\NormalTok{)}
\KeywordTok{plot}\NormalTok{(x,}\KeywordTok{dnorm}\NormalTok{(x,}\DecValTok{2}\OperatorTok{/}\DecValTok{4}\NormalTok{,}\KeywordTok{sqrt}\NormalTok{(}\DecValTok{2}\OperatorTok{/}\DecValTok{4}\NormalTok{)),}\DataTypeTok{type=}\StringTok{"l"}\NormalTok{,}\DataTypeTok{ylab=}\StringTok{"f(x)"}\NormalTok{,}\DataTypeTok{main=}\StringTok{"N(2/4,√2/4)"}\NormalTok{)}
\end{Highlighting}
\end{Shaded}

\includegraphics{Practica3_ÁlvaroLarraya_files/figure-latex/unnamed-chunk-17-1.pdf}

¿Que función de densidad se ajusta mejor al histográma, la gamma o la
normal?

La función gamma se ajusta mejor, ya que en la normal se representan x
con valores negativos los cuales no tienen mucho sentido en este
contexto, ya que no podemos hablar de la muestra i-ésima negativa. Las
muestras, luego las x solo toman valores naturales.

2.5) Repite el experimento con muestras de tamaño n=10, 20, y 100. Fija
la semilla aleatoria a 235.

\hypertarget{n10}{%
\paragraph{n=10}\label{n10}}

\begin{Shaded}
\begin{Highlighting}[]
\KeywordTok{set.seed}\NormalTok{(}\DecValTok{235}\NormalTok{)}
\NormalTok{m<-}\DecValTok{20000}
\NormalTok{n}\OperatorTok{<}\DecValTok{10}
\end{Highlighting}
\end{Shaded}

\begin{verbatim}
## [1] TRUE
\end{verbatim}

\begin{Shaded}
\begin{Highlighting}[]
\NormalTok{sum.exp <-}\StringTok{ }\KeywordTok{numeric}\NormalTok{(m)}
\ControlFlowTok{for}\NormalTok{(i }\ControlFlowTok{in} \DecValTok{1}\OperatorTok{:}\NormalTok{m)\{}
\NormalTok{samp <-}\StringTok{ }\KeywordTok{rexp}\NormalTok{(n,}\DecValTok{4}\NormalTok{)}
\NormalTok{sum.exp[i] <-}\StringTok{ }\KeywordTok{sum}\NormalTok{(samp)}
\NormalTok{\}}
\KeywordTok{hist}\NormalTok{(sum.exp,}\DataTypeTok{prob=}\OtherTok{TRUE}\NormalTok{, }\DataTypeTok{main=}\StringTok{"Suma de muestras"}\NormalTok{)}
\NormalTok{x <-}\StringTok{ }\KeywordTok{seq}\NormalTok{(}\DecValTok{0}\NormalTok{,}\DecValTok{10}\NormalTok{,}\FloatTok{0.01}\NormalTok{)}
\NormalTok{y <-}\StringTok{ }\KeywordTok{dexp}\NormalTok{(x,}\DecValTok{4}\NormalTok{)}
\KeywordTok{lines}\NormalTok{(x,y)}
\end{Highlighting}
\end{Shaded}

\includegraphics{Practica3_ÁlvaroLarraya_files/figure-latex/unnamed-chunk-18-1.pdf}

Añade la función de densidad de una Γ(α = 10, λ = 4)

\begin{Shaded}
\begin{Highlighting}[]
\NormalTok{x<-}\KeywordTok{seq}\NormalTok{(}\DecValTok{0}\NormalTok{,}\DecValTok{10}\NormalTok{,}\FloatTok{0.01}\NormalTok{)}
\KeywordTok{plot}\NormalTok{(x,}\KeywordTok{dgamma}\NormalTok{(x,}\DecValTok{10}\NormalTok{,}\DecValTok{4}\NormalTok{),}\DataTypeTok{type=}\StringTok{"l"}\NormalTok{,}\DataTypeTok{ylab=}\StringTok{"f(x)"}\NormalTok{,}\DataTypeTok{main=}\StringTok{"Γ(α = 10, λ = 4)"}\NormalTok{)}
\end{Highlighting}
\end{Shaded}

\includegraphics{Practica3_ÁlvaroLarraya_files/figure-latex/unnamed-chunk-19-1.pdf}

Y una N(10/4,√10/4)

\begin{Shaded}
\begin{Highlighting}[]
\NormalTok{x<-}\KeywordTok{seq}\NormalTok{(}\OperatorTok{-}\DecValTok{3}\NormalTok{,}\DecValTok{8}\NormalTok{,}\FloatTok{0.01}\NormalTok{)}
\KeywordTok{plot}\NormalTok{(x,}\KeywordTok{dnorm}\NormalTok{(x,}\DecValTok{10}\OperatorTok{/}\DecValTok{4}\NormalTok{,}\KeywordTok{sqrt}\NormalTok{(}\DecValTok{10}\OperatorTok{/}\DecValTok{4}\NormalTok{)),}\DataTypeTok{type=}\StringTok{"l"}\NormalTok{,}\DataTypeTok{ylab=}\StringTok{"f(x)"}\NormalTok{,}\DataTypeTok{main=}\StringTok{"N(10/4,√10/4)"}\NormalTok{)}
\end{Highlighting}
\end{Shaded}

\includegraphics{Practica3_ÁlvaroLarraya_files/figure-latex/unnamed-chunk-20-1.pdf}

\hypertarget{n20}{%
\paragraph{n=20}\label{n20}}

\begin{Shaded}
\begin{Highlighting}[]
\KeywordTok{set.seed}\NormalTok{(}\DecValTok{235}\NormalTok{)}
\NormalTok{m<-}\DecValTok{20000}
\NormalTok{n}\OperatorTok{<}\DecValTok{20}
\end{Highlighting}
\end{Shaded}

\begin{verbatim}
## [1] TRUE
\end{verbatim}

\begin{Shaded}
\begin{Highlighting}[]
\NormalTok{sum.exp <-}\StringTok{ }\KeywordTok{numeric}\NormalTok{(m)}
\ControlFlowTok{for}\NormalTok{(i }\ControlFlowTok{in} \DecValTok{1}\OperatorTok{:}\NormalTok{m)\{}
\NormalTok{samp <-}\StringTok{ }\KeywordTok{rexp}\NormalTok{(n,}\DecValTok{4}\NormalTok{)}
\NormalTok{sum.exp[i] <-}\StringTok{ }\KeywordTok{sum}\NormalTok{(samp)}
\NormalTok{\}}
\KeywordTok{hist}\NormalTok{(sum.exp,}\DataTypeTok{prob=}\OtherTok{TRUE}\NormalTok{, }\DataTypeTok{main=}\StringTok{"Suma de muestras"}\NormalTok{)}
\NormalTok{x <-}\StringTok{ }\KeywordTok{seq}\NormalTok{(}\DecValTok{0}\NormalTok{,}\DecValTok{10}\NormalTok{,}\FloatTok{0.01}\NormalTok{)}
\NormalTok{y <-}\StringTok{ }\KeywordTok{dexp}\NormalTok{(x,}\DecValTok{4}\NormalTok{)}
\KeywordTok{lines}\NormalTok{(x,y)}
\end{Highlighting}
\end{Shaded}

\includegraphics{Practica3_ÁlvaroLarraya_files/figure-latex/unnamed-chunk-21-1.pdf}

Añade la función de densidad de una Γ(α = 20, λ = 4)

\begin{Shaded}
\begin{Highlighting}[]
\NormalTok{x<-}\KeywordTok{seq}\NormalTok{(}\DecValTok{0}\NormalTok{,}\DecValTok{10}\NormalTok{,}\FloatTok{0.01}\NormalTok{)}
\KeywordTok{plot}\NormalTok{(x,}\KeywordTok{dgamma}\NormalTok{(x,}\DecValTok{20}\NormalTok{,}\DecValTok{4}\NormalTok{),}\DataTypeTok{type=}\StringTok{"l"}\NormalTok{,}\DataTypeTok{ylab=}\StringTok{"f(x)"}\NormalTok{,}\DataTypeTok{main=}\StringTok{"Γ(α = 20, λ = 4)"}\NormalTok{)}
\end{Highlighting}
\end{Shaded}

\includegraphics{Practica3_ÁlvaroLarraya_files/figure-latex/unnamed-chunk-22-1.pdf}

Y una N(20/4,√20/4)

\begin{Shaded}
\begin{Highlighting}[]
\NormalTok{x<-}\KeywordTok{seq}\NormalTok{(}\DecValTok{0}\NormalTok{,}\DecValTok{10}\NormalTok{,}\FloatTok{0.01}\NormalTok{)}
\KeywordTok{plot}\NormalTok{(x,}\KeywordTok{dnorm}\NormalTok{(x,}\DecValTok{20}\OperatorTok{/}\DecValTok{4}\NormalTok{,}\KeywordTok{sqrt}\NormalTok{(}\DecValTok{20}\OperatorTok{/}\DecValTok{4}\NormalTok{)),}\DataTypeTok{type=}\StringTok{"l"}\NormalTok{,}\DataTypeTok{ylab=}\StringTok{"f(x)"}\NormalTok{,}\DataTypeTok{main=}\StringTok{"N(20/4,√20/4)"}\NormalTok{)}
\end{Highlighting}
\end{Shaded}

\includegraphics{Practica3_ÁlvaroLarraya_files/figure-latex/unnamed-chunk-23-1.pdf}

\hypertarget{n10-1}{%
\paragraph{n=10}\label{n10-1}}

\begin{Shaded}
\begin{Highlighting}[]
\KeywordTok{set.seed}\NormalTok{(}\DecValTok{235}\NormalTok{)}
\NormalTok{m<-}\DecValTok{20000}
\NormalTok{n}\OperatorTok{<}\DecValTok{100}
\end{Highlighting}
\end{Shaded}

\begin{verbatim}
## [1] TRUE
\end{verbatim}

\begin{Shaded}
\begin{Highlighting}[]
\NormalTok{sum.exp <-}\StringTok{ }\KeywordTok{numeric}\NormalTok{(m)}
\ControlFlowTok{for}\NormalTok{(i }\ControlFlowTok{in} \DecValTok{1}\OperatorTok{:}\NormalTok{m)\{}
\NormalTok{samp <-}\StringTok{ }\KeywordTok{rexp}\NormalTok{(n,}\DecValTok{4}\NormalTok{)}
\NormalTok{sum.exp[i] <-}\StringTok{ }\KeywordTok{sum}\NormalTok{(samp)}
\NormalTok{\}}
\KeywordTok{hist}\NormalTok{(sum.exp,}\DataTypeTok{prob=}\OtherTok{TRUE}\NormalTok{, }\DataTypeTok{main=}\StringTok{"Suma de muestras"}\NormalTok{)}
\NormalTok{x <-}\StringTok{ }\KeywordTok{seq}\NormalTok{(}\DecValTok{0}\NormalTok{,}\DecValTok{10}\NormalTok{,}\FloatTok{0.01}\NormalTok{)}
\NormalTok{y <-}\StringTok{ }\KeywordTok{dexp}\NormalTok{(x,}\DecValTok{4}\NormalTok{)}
\KeywordTok{lines}\NormalTok{(x,y)}
\end{Highlighting}
\end{Shaded}

\includegraphics{Practica3_ÁlvaroLarraya_files/figure-latex/unnamed-chunk-24-1.pdf}

Añade la función de densidad de una Γ(α = 20, λ = 4)

\begin{Shaded}
\begin{Highlighting}[]
\NormalTok{x<-}\KeywordTok{seq}\NormalTok{(}\DecValTok{15}\NormalTok{,}\DecValTok{35}\NormalTok{,}\FloatTok{0.01}\NormalTok{)}
\KeywordTok{plot}\NormalTok{(x,}\KeywordTok{dgamma}\NormalTok{(x,}\DecValTok{100}\NormalTok{,}\DecValTok{4}\NormalTok{),}\DataTypeTok{type=}\StringTok{"l"}\NormalTok{,}\DataTypeTok{ylab=}\StringTok{"f(x)"}\NormalTok{,}\DataTypeTok{main=}\StringTok{"Γ(α = 100, λ = 4)"}\NormalTok{)}
\end{Highlighting}
\end{Shaded}

\includegraphics{Practica3_ÁlvaroLarraya_files/figure-latex/unnamed-chunk-25-1.pdf}

Y una N(20/4,√20/4)

\begin{Shaded}
\begin{Highlighting}[]
\NormalTok{x<-}\KeywordTok{seq}\NormalTok{(}\DecValTok{10}\NormalTok{,}\DecValTok{40}\NormalTok{,}\FloatTok{0.01}\NormalTok{)}
\KeywordTok{plot}\NormalTok{(x,}\KeywordTok{dnorm}\NormalTok{(x,}\DecValTok{100}\OperatorTok{/}\DecValTok{4}\NormalTok{,}\KeywordTok{sqrt}\NormalTok{(}\DecValTok{100}\OperatorTok{/}\DecValTok{4}\NormalTok{)),}\DataTypeTok{type=}\StringTok{"l"}\NormalTok{,}\DataTypeTok{ylab=}\StringTok{"f(x)"}\NormalTok{,}\DataTypeTok{main=}\StringTok{"N(100/4,√100/4)"}\NormalTok{)}
\end{Highlighting}
\end{Shaded}

\includegraphics{Practica3_ÁlvaroLarraya_files/figure-latex/unnamed-chunk-26-1.pdf}

¿Qué observas?

Podemos ver que la varianza de la normal y la gamma aumentan y desplazan
las x con mayor probabilidad hacia la derecha en la gráfica (los valores
de x con mayor probabilidad toman mayores valores) cada vez que se
aumenta el tamaño de las muestras. Mientras que la gráfica de la suma de
las muestras se mantiene constante al variar el tamaño de estas (las
muestras).

\hypertarget{verifica-empuxedricamente-la-afirmaciuxf3n-propuesta}{%
\subsubsection{2.6) Verifica empíricamente la afirmación
propuesta}\label{verifica-empuxedricamente-la-afirmaciuxf3n-propuesta}}

Sabiendo que Z = N(0,1). Represento la formula de F

\begin{Shaded}
\begin{Highlighting}[]
\NormalTok{x<-}\KeywordTok{seq}\NormalTok{(}\OperatorTok{-}\DecValTok{100}\NormalTok{,}\DecValTok{40}\NormalTok{,}\FloatTok{0.01}\NormalTok{)}
\KeywordTok{plot}\NormalTok{(x,((}\KeywordTok{dnorm}\NormalTok{(x,}\DecValTok{0}\NormalTok{,}\DecValTok{1}\NormalTok{)}\OperatorTok{^}\DecValTok{2}\NormalTok{)}\OperatorTok{/}\NormalTok{(}\KeywordTok{dchisq}\NormalTok{(x,}\DecValTok{20}\NormalTok{)}\OperatorTok{/}\DecValTok{20}\NormalTok{)),}\DataTypeTok{type=}\StringTok{"l"}\NormalTok{,}\DataTypeTok{ylab=}\StringTok{"f(x)"}\NormalTok{,}\DataTypeTok{main=}\StringTok{"F"}\NormalTok{)}
\end{Highlighting}
\end{Shaded}

\includegraphics{Practica3_ÁlvaroLarraya_files/figure-latex/unnamed-chunk-27-1.pdf}

Comparo con la gráfica de la aproximación F 1,20

\begin{Shaded}
\begin{Highlighting}[]
\NormalTok{x<-}\KeywordTok{seq}\NormalTok{(}\OperatorTok{-}\DecValTok{100}\NormalTok{,}\DecValTok{40}\NormalTok{,}\FloatTok{0.01}\NormalTok{)}
\KeywordTok{plot}\NormalTok{(x,}\KeywordTok{df}\NormalTok{(x,}\DecValTok{1}\NormalTok{,}\DecValTok{20}\NormalTok{),}\DataTypeTok{type=}\StringTok{"l"}\NormalTok{,}\DataTypeTok{ylab=}\StringTok{"f(x)"}\NormalTok{,}\DataTypeTok{main=}\StringTok{"F 1,20"}\NormalTok{)}
\end{Highlighting}
\end{Shaded}

\includegraphics{Practica3_ÁlvaroLarraya_files/figure-latex/unnamed-chunk-28-1.pdf}

Comprobamos así que efectivamente se da esa equivalencia
(aproximadamente), para valores positivos de F

\end{document}
